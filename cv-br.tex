% cv-br.tex
%   by alemedeiros <alexandre.n.medeiros _at_ gmail.com>
%
% my curriculum vitae -- brazilian portuguese version
%
% License:
% % CC BY-NC-SA 3.0 (http://creativecommons.org/licenses/by-nc-sa/3.0/)
%
% created: 2013-05-26
% last revision: 2016-04-20

\documentclass[10pt,a4paper,sans]{moderncv}
\usepackage{fullpage}
\usepackage[brazilian]{babel}
\usepackage[utf8]{inputenc}
\usepackage[T1]{fontenc}
\usepackage{roboto}

% cv style
\moderncvtheme[purple]{casual}

% personal info
\firstname{Alexandre}
\familyname{Medeiros}

% optional info
\title{Curriculum Vitae}
\mobile{+55~(11)~96624~2947}
\email{alexandre.n.medeiros@gmail.com}
\homepage{http://alemedeiros.sdf.org}
\social[github]{alemedeiros}

\begin{document}
\makecvtitle{}
\section{Resumo}{Um programador versátil com distinto desempenho acadêmico. Interessado por diversas áreas da computação variando desde programação em baixo nível e Sistemas Operacionais até conceitos abstratos e de alto nívels como Programação Funcional e o modelo MapReduce de programação. Eu sinto prazer em aprender novas habilidades e ao enfrentar desafios, meu objetivo é nunca deixar de crescer profissionalmente e pessoalmente.}

\section{Educação}
\cventry{2011 --- presente}{Bacharelado em Ciência da Computação}{Universidade Estadual de Campinas}{Campinas, São Paulo, Brasil}{Em Andamento (Previsão de graduação: Dezembro de 2016)}{%
	Coeficiente de rendimento atual: 8,6/10\newline
	Prêmio Pascal 2014 (Melhor desempenho dentre os alunos de Ciência da Computação)
}
\cventry{Setembro 2014 --- Agosto 2015}{Bacharelado em Ciência da Computação}{Queen Mary University of London}{Londres, Reino Unido}{}{Estudante associado (Financiado por bolsa do governo brasileiro: Ciência sem Fronteiras)}

\section{Experiência}

\subsection{Profissional}

\cventry{Dezembro 2015 --- Março 2016}{Health Search Exploration}{Estagiário em Engenharia de Software}{Google}{Belo Horizonte, Minas Gerais, Brasil}{%
}

\subsection{Acadêmica}

\cventry{2013 --- 2014}{Consumo de energia em dispositivos móveis ARM}{Iniciação Científica}{Universidade Estadual de Campinas}{Campinas, São Paulo, Brasil}{%
	Orientador: Prof.\ Edson Borin (IC-Unicamp)\newline
	Co-Orientador: Prof.\ Sandro Rigo (IC-Unicamp)\newline
	Financiado por bolsa da Samsung
}

\cventry{2013}{Competição Intel de Sistemas Embarcados 2013}{2º lugar}{III Simpósio Brasileiro de Engenharia de Sistemas Computacionais}{Niterói, Rio de Janeiro, Brasil}{%
	Grupo: Gabriel Krisman Bertazi, Augusto dos Santos Morgan e Alexandre Medeiros\newline
	Orientadores: Prof.\ Siome Goldenstein e Prof.\ Edson Borin
}

\cventry{2012 --- 2016}{Monitoria}{Programa de Apoio Didático}{Universidade Estadual de Campinas}{Campinas, São Paulo, Brasil}{%
	\begin{itemize}
		\item Programação Orientada a Objetos --- 1º semestre de 2016
		\item Algoritmos e Programação de Computadores --- 1º semestre de 2014
		\item Algoritmos e Programação de Computadores --- 2º semestre de 2013
		\item Organização Básica de Computadores e Linguagem de Montagem --- 1º semestre de 2013
		\item Estruturas de Dados --- 2º semestre de 2012
		\item Estruturas de Dados --- 1º semestre de 2012
	\end{itemize}
}

\subsection{Voluntária}

\cventry{2011 --- 2014}{Coordenador Administrativo}{CACo (Centro Acadêmico da Computação)}{Universidade Estadual de Campinas}{Campinas, São Paulo, Brasil}{Centro acadêmico dos estudantes de computação da Unicamp}

\cventry{2012 --- 2013}{Representante Discente de Graduação}{Instituto de Computação}{Universidade Estadual de Campinas}{Campinas, São Paulo, Brasil}{Representante Discente na Comissão de Graduação do Instituto de Computação durante o período de 2012--13}

\cventry{2012, 2013}{Instrutor de GNU/Linux básico para Ingressantes}{GPSL}{Universidade Estadual de Campinas}{Campinas, São Paulo, Brasil}{Aulas básicas de GNU/Linux para Alunos de Computação}

\section{Conhecimento Técnico}
\cvitem{Básico}{ARM Assembly, VHDL, Go, Lua, JavaScript, PHP, SQL, Django Web Framework}
\cvitem{Intermediário}{Haskell, Sistemas operacionais GNU/Linux, Programação Paralela, Shell Script, git, \LaTeX{}, Java}
\cvitem{Avançado}{C/C++, Python}

\section{Idiomas}
\cvlanguage{Português Brasileiro}{Fluente}{Língua Nativa}
\cvlanguage{Inglês}{Fluente}{Média de 8.0 no teste IELTS Academic (2013)}

\end{document}
